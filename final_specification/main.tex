\documentclass{article}
\usepackage{tikz}
\usepackage{todonotes}
\usetikzlibrary{arrows,decorations.pathmorphing}

\hyphenation{Schwell-werte}
% =============== Definitionen =========
\newenvironment{definition}
    [1]
    {
        {\b Definition:} #1\\
    }
    {}

\newcommand{\defined}
    [1]
    {
        {\bf #1}
    }
% ======================================

% =============== PHASES ===============
\newcounter{ycounter}{}
\newenvironment{phases}
{
\setcounter{ycounter}{-1}
\begin{tikzpicture}[node distance=3cm,text width=3cm,
    box/.style={shape=rectangle,draw}]
}
{\end{tikzpicture}}

\newcommand{\nomoreheaders} {
    \addtocounter{ycounter}{-1}
}

\newcommand{\aliceheader}[1] {
\node at (-4,\theycounter) {#1}
}
\newcommand{\bobheader}[1] {
\node at (4,\theycounter) {#1}
}

\newcommand{\aliceknowledge}[2][0]{
\node[box]
    at (-4+#1, \theycounter)
    {#2}
}
\newcommand{\bobknowledge}[2][0]{
\node[box]
    at (4+#1, \theycounter)
    {#2}
}

\newcommand{\symmetricknowledge}[2][0]{
\aliceknowledge{#2};
\bobknowledge[#1]{#2};
}


\newcommand{\phase}[1]{
\addtocounter{ycounter}{-1};
\draw[->,decorate,decoration={snake,amplitude=1mm,post length=2mm}]
    (0,\theycounter) -- ++(0,-2)
    node [right,midway,xshift=3mm] {#1};
\addtocounter{ycounter}{-3};
}
% ======================================

\begin{document}
\listoftodos
\pagebreak
\tableofcontents
\pagebreak
\section{Einleitung}
\todo[inline]{Struktur vom Dokument erl\"autern}
\subsection{Begriffe}
\begin{definition}{Eigenes,Gesamtes}
\(M\) sei eine Menge von Elementen, die in zwei Teilmengen \(M_A\)
und \(M_B\) zerf\"allt, sodass \(M = M_A \cup M_B\) ist. Wir nehmen
desweiteren an, dass Alice \(M_A\) kennt, aber weder \(M\) noch \(M_B\)
und dass Bob \(M_B\) kennt, aber weder \(M\) noch \(M_A\). Dann bezeichnen
wir:
\begin{itemize}
\item \(M\) als \defined{gesamtes} Wissen
\item \(M_A\) als das \defined{eigene} Wissen von Alice
\item \(M_B\) als das \defined{eigene} Wissen von Bob
\item \(M_B\) als das \defined{andere} Wissen von Alice
\item \(M_A\) als das \defined{andere} Wissen von Bob
\end{itemize}
\end{definition}
\todo[inline]{Definition: 'Gemeinsam' := "etwas, was beide anwender haben und gleich ist"}
\todo[inline]{Definition: Vorwissen := "Wissen aus vorherigen Phasen"}
\todo[inline]{Definition: Entscheidungsbaum}
\todo[inline]{Definition: Attribut}
\todo[inline]{Definition: Schaltkreis}
\todo[inline]{Definition: Entstellter Schaltkreis}
\subsection{Annahmen}
\todo[inline]{Annahme: ehrliche anwender := "handeln nach protokoll"}

\section{Grundlagen der Anwendung}
\todo[inline]{Vision der Anwendung}

\subsection{Form der Benutzereingabe}
\todo[inline]{Festlegen, wie die E-Mails ins Programm kommen}

\subsection{Interaktion der verteilten Programme}
\todo[inline]{Festlegen, wie das Programm verteilt wird und die Teile kommunizieren}

\subsection{Phasen der Anwendung}
\todo[inline]{Kurz die Einzelnen phasen der Anwendung beschreiben}
\begin{figure}[htb]
\centering
\begin{phases}
\aliceheader{Alice};
\bobheader{Bob};
\nomoreheaders
\aliceknowledge[-2]{eigene E-Mails};
\bobknowledge[-2]{eigene E-Mails};

\aliceknowledge[2]{Einordnung von eigenen E-Mails als Spam, Nicht Spam};
\bobknowledge[2]{Einordnung von eigenen E-Mails als Spam, Nicht Spam};

\phase{Phase 1: Finden der gemeinsamen Wortliste};

\symmetricknowledge{Gemeinsame Wortliste};

\phase{Phase 2: Finden der gemeinsamen Schwellwerte};

\symmetricknowledge{Gemeinsame Attribute = Liste von (Wort + Schwellwerte)};
\phase{Phase 3: Diskretisieren der eigenene E-Mails};

\symmetricknowledge{Eigene diskretisierte E-Mails};

\phase{Phase 4: Lernen der gesamten E-Mails};

\symmetricknowledge{Gemeinsamer Klassifikator};
\end{phases}
\caption{Phasen der Anwendung}
\end{figure}


\pagebreak % XXX: if possible, kill this pagebreak
\section{Finden der gemeinsamen Wortliste}
\todo[inline]{Einleitung, Verweisen auf Figure f\"ur gemeinsame Wortliste}
\begin{figure}[htb]
\begin{phases}
\aliceheader{Alice};
\bobheader{Bob};
\nomoreheaders

\aliceknowledge[-2]{eigene E-Mails};
\bobknowledge[-2]{eigene E-Mails};

\aliceknowledge[2]{Einordnung von eigenen E-Mails als Spam, Nicht Spam};
\bobknowledge[2]{Einordnung von eigenen E-Mails als Spam, Nicht Spam};
\phase{Phase 1.1: Berechnung der Vorkommnisse der Worte in eigenen Spam/Nicht Spam E-Mails};
\symmetricknowledge[1]{Liste von (Wort + Anteil der Vorkomnisse eines Wortes an allen eigenen Worten)};
\phase{Phase 1.2: Sortierung der eigenen Worte nach Informationsheuristik, Auswahl der N besten eigenen Worten};
\symmetricknowledge{Eigene Wortliste}
\phase{Phase 1.3: Synchronisierung der Wortlisten}
\symmetricknowledge{Gemeinsame Wortliste}
\end{phases}
\caption{Schritte zum Berechnen der gemeinsamen Wortliste}
\end{figure}

\todo[inline]{F\"ur section-Titel besseren Begriff f\"ur "Vorkomnisse der Worte in eigenen Spam/Nicht Spam E-Mails" finden}
\subsection{Berechnung der Vorkomnisse}
\todo[inline]{Content}
\subsection{Sortierung der Worte nach Informationsheuristik}
\todo[inline]{Content}
\subsection{Syncronisierung der Wortlisten}
\todo[inline]{Content}

\pagebreak % XXX: if possible, kill this pagebreak
\section{Finden der gemeinsamen Schwellwerte}
\todo[inline]{Einleitung, auf Figure f\"ur Schwellwerte verweisen}

\begin{figure}[htb]
\begin{phases}
\aliceheader{Alice};
\bobheader{Bob};
\nomoreheaders

\aliceknowledge[-2]{Vorwissen};
\aliceknowledge[2]{Gemeinsame Wortliste};

\bobknowledge[-2]{Vorwissen};
\bobknowledge[2]{Gemeinsame Wortliste};
\phase{Phase 2.1: Berechnung der Vorkommnisse der Worte in eigenen Spam/Nicht Spam E-Mails};
\symmetricknowledge[1]{Eigene Liste von (Wort + Anteil der Vorkomnisse eines Wortes an Spam/Nicht-Spam Worten)};
\phase{Phase 2.2: Bestimmung eines Schwellwertes, der Spam, Nicht-Spam Anteile m\"oglichst halbiert};
\symmetricknowledge{Eigene Liste von (Wort + Schwellwert)};
\phase{Phase 2.3: Synchronisierung der Schwellwerte};
\symmetricknowledge{Gemeinsame Liste von (Wort + Schwellwerte)};
\end{phases}
\end{figure}
\todo[inline]{F\"ur section-Titel besseren Begriff f\"ur "Vorkomnisse der Worte in eigenen Spam/Nicht Spam E-Mails" finden}
\subsection{Berechnung der Vorkomnisse}
\todo[inline]{Content}
\subsection{Bestimmung der eigenen Schwellwerte}
\todo[inline]{Content}
\subsection{Syncronisierung der Schwellwerte}
\todo[inline]{Content}

\pagebreak % XXX: if possible, kill this pagebreak
\section{Diskretisieren der eigenen E-Mails}
\todo[inline]{Einleitung, Figure referenzieren}
\todo[inline]{Content}
\begin{figure}[htb]
\begin{phases}
\aliceheader{Alice};
\bobheader{Bob};
\nomoreheaders;
\aliceknowledge[-2]{Vorwissen};
\aliceknowledge[2]{Gemeinsame Attribute = Liste von (Wort + Schwellwerte)};

\bobknowledge[-2]{Vorwissen};
\bobknowledge[2]{Gemeinsame Attribute = Liste von (Wort + Schwellwerte)};

\phase{Phase 3: Diskretisieren der eigenene E-Mails};
\symmetricknowledge{Eigene diskretisierte E-Mails};
\end{phases}
\end{figure}

\pagebreak % XXX: if possible, kill this pagebreak
\section{Lernen der gesamten E-Mails}
\todo[inline]{verteiltes ID3 beschreiben}

\subsection{Yaos Protokoll}
\begin{figure}[htb]
\begin{phases}
\aliceheader{Alice};
\bobheader{Bob};
\nomoreheaders
\symmetricknowledge{Eigene Eingabe};
\phase{Y.1: Konstruktion des entstellten Schaltkreises und \"Ubertragung};
\aliceknowledge[-2]{Eingabekodierung};
\aliceknowledge[2]{Schaltkreis};
\bobknowledge{Schaltkreis};
\phase{Y.3 Alice schickt kodierte Eingabe an Bob};
\bobknowledge{Alice' kodierte Eingabe};
\phase{Y.4 Via Oblivious Transfer wird Bob's Eingabe kodiert};
\bobknowledge{Bob's kodierte Eingabe};
\phase{Y.5 Bob wertet Schaltkreis aus und sendet die Ausgabe an Alice};
\symmetricknowledge{Ausgabe} ;
\end{phases}
\caption{Yaos Algorithmus}
\end{figure}
\todo[inline]{Yaos algorithmus grundlegend zusammenfassen (garbled decision table, garbled gate, garbled circuit)}
\todo[inline]{Verschl\"usselung f\"ur die garbled Circuits festlegen (RSA?)}
\todo[inline]{1-2 Oblivious Transfer festlegen (mit RSA?)}
\todo[inline]{Feststellung der benoetigten Bytezahl beschreiben}


\subsection{Feststellen der dominierenden Ausgabe}
\begin{figure}[htb]
\begin{phases}
\aliceknowledge[-2]{Vorwissen};
\aliceknowledge[2]{Eigene diskretisierte E-Mails};

\bobknowledge[-2]{Vorwissen};
\bobknowledge[2]{Eigene diskretisierte E-Mails};

\phase{Phase 4.1: Feststellen der dominierenden Ausgabe};

\symmetricknowledge{gemeinsamer Blattknoten mit dominierender Ausgabe};
\end{phases}
\caption{ID3-Algorithmus, Fall 1: Keine Attribute mehr vorhanden}
\end{figure}
\todo[inline]{Schaltkreis designen: Maximum von Summen von positiven Zahlensequenzen}

\subsection{Feststellen ob Ausgabe eindeutig}
\begin{figure}[htb]
\begin{phases}
\aliceknowledge[-2]{Vorwissen};
\aliceknowledge[2]{Eigene diskretisierte E-Mails};

\bobknowledge[-2]{Vorwissen};
\bobknowledge[2]{Eigene diskretisierte E-Mails};

\phase{Phase 4.2: Feststellen ob Ausgabe eindeutig}

\symmetricknowledge{gemeinsamer Blattknoten mit eindeutiger Ausgabe}
\end{phases}
\caption{ID3-Algorithmus, Fall 2: Ausgabe eindeutig}
\end{figure}
\todo[inline]{Schaltkreis designen: Gleichheit.}

\subsection{Das Entropien-Protokoll}
\todo[inline]{Schaltkreis f\"ur x * log x -Protokoll aus dem Paper zusammenfassen}

\subsection{Attribut mit maximalem Informationsgewinn finden}
\begin{figure}[htb]
\begin{phases}
\aliceknowledge[-2]{Vorwissen};
\aliceknowledge[2]{Eigene diskretisierte E-Mails};

\bobknowledge[-2]{Vorwissen};
\bobknowledge[2]{Eigene diskretisierte E-Mails};

\phase{Phase 4.3: Berechnen der Entropien}

\symmetricknowledge{Gemeinsame Entropien}

\phase{Phase 4.4: Attribut mit maximalem Informationsgewinn finden}

\symmetricknowledge{Gemeinsames bestes aktuelles Attribut}

\phase{Phase 4: Rekursion}

\symmetricknowledge{gemeinsamer Klassifikator}
\end{phases}
\caption{ID3-Algorithmus, Fall 3: Erzeugung eines Astes}
\end{figure}
\todo[inline]{Vorgheen zusammenfassen, Schaltkreis aus dominierender Ausgabe wiederverwenden}

\pagebreak  %XXX: if  possible, kill this pagebreak
\section{Verwenden des Klassifikators}
\todo[inline]{Einleitung: Wir brauchen ein Programm, was den Klassifkator auf eine MAil oder Mails anwendet}
\subsection{Eingabe des Klassifikators}
\todo[inline]{Anhand der Definition von Attributen und Entscheidungsb\"aumen beschreibungssprache fuer Entscheidungsbaum herleiten}
\subsection{Arbeitsweise des Klassifikators}
\todo[inline]{Arbeitsweise des Klassifikators erkl\"aren}
\end{document}
