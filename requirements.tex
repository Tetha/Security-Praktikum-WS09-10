\documentclass{article}
\usepackage{color}
\usepackage{multicol}
% =============== Requirements =========
\newcounter{requirementscount}{}
\setcounter{requirementscount}{0}
\newcommand{\requirement}[1] {
        \addtocounter{requirementscount}{1}
        {\bf Requirement \therequirementscount:} #1\\
    }

\begin{document}
\section{Lerner}
\requirement{Das Programm liest die E-Mails aus dem
Verzeichnis, das der Benutzer angegeben hat, ein und 
behandelt sie entsprechend der Unterverzeichnisse als Spam
oder Nicht Spam}
\requirement{Wenn der Anwender keine Aussage \"uber die Ausgabeform des
Programmes trifft, wird der Klassifikator auf die Standardausgabe geschrieben}\\
\requirement{Wenn der Anwender mit einem Kommandozeilenparameter einen
Dateinamen f\"ur die Ausgabe definiert, wird der Klassifikator in diese
Ausgabedatei geschrieben.}\\
\requirement{Wenn der Anwender mit einem Kommandozeilenparameter
einen Dateinamen f\"ur die Ausgabe definiert und dieser Dateiname
ist `-', dann wird der Klassifikator auf die Standardausgabe 
geschrieben}\\
\requirement{Wenn der Anwender den Ausgabeparameter mehrfach angibt,
gibt die Anwendung einen Fehler aus}\\
\requirement{Die Anwendung kann mit einem Parameter \texttt{--server} gestartet
werden. Zus\"atzlich muss ein Parameter \texttt{--port=INTEGER} angegeben werden.
Dies veranlasst die Anwendung, einen Server zu starten, welcher auf dem
Rechner auf dem angegebenen Port auf den Client wartet. Falls hier ein
Fehler auftritt (beispielsweise bei Berechtigungsproblemen oder bei
bereits belegten Ports) bricht der Server mit einer entsprechenden Fehlermeldung
ab}
\requirement{Die Anwendung kann mit einem Parameter \texttt{--client} gestartet
werden. Zus\"atzlich m\"ussen dann die Parameter \texttt{--port=INTEGER}
und \texttt{--server-ip=IP-Addresse} angegeben sein. Die Anwendung versucht
dann, eine Verbindung zum Server unter der gegebenen IP und dem gegebenen
Port aufzubauen und den Lernvorgang durchzuf\"uhren. Sowohl Client als
auch Server beenden sich nach einem Lernvorgang. Sollte beim Verbinden
ein Problem auftreten (Unerreichbarkeit des Servers, Port geschlossen),
so beendet sich der Client mit einer entsprechenden Fehlermeldung}\\
\requirement{Wenn die Spam-Mails die Mails "Foo Bar", "Foo Bar", "Foo Foo"
und "Foo" sind und die Nicht-Spam-Mails "Bar Bar", "Foo" und "Bar", 
dann muss diese Phase die folgende Tabelle berechnen:\\
\begin{center}
\begin{tabular}{c c c}
Wort & Spam-Anteil & Nicht-Spam-Anteil \\
Foo & \(\frac{5}{7}\) & \(\frac{1}{4}\) \\
Bar & \(\frac{2}{7}\) & \(\frac{3}{4}\)
\end{tabular}
\end{center}
Eine Approximation der Werte durch Fliesskommazahlen ist
ebenfalls akzeptabel.}
\requirement{Wenn N = 2 ist, und als Worte mit Vorkomnissen gegeben sind:
\begin{center}
\begin{tabular}{c c c}
Wort & Vorkomnisse in Spam-E-Mails & Vorkommnisse in Nicht-Spam-Emails \\
A & 0.5 & 0.5 \\
B & 0.2 & 0.2 \\
C & 0.3 & 0.5 \\
D & 0.2 & 0.8 \\
E & 0.9 & 0.2
\end{tabular}
\end{center}
dann werden als Wortliste D und E gew\"ahlt.}
\requirement{Wenn Alice die Wortliste A, B, C berechnet hat, und Bob die
Wortliste C, D, E berechnet hat, dann muss die zusammengefasste Wortliste
A, B, C, D, E sein.} 

\requirement{Das Protokoll, in dem beide Anwender ihre Wortlisten einander
zusenden und lokal die spezifizierte Syncronisierung durchf\"uhren, ist
implementiert}
\requirement{Wenn die E-Mails "A A A", "A B B", "A C C" und "A A C" sind, dann
muss die Software als eigenen Mittelwert f\"ur das Wort A 
\(\frac{1}{4} \cdot (1 + \frac{1}{3} + \frac{1}{3} + \frac{2}{3}) = \frac{7}{12}\) bestimmen.}
\requirement{Wenn a = 0.2 ist und b = 0.4, dann ist das Schwellwertpaar
(0.2, 0.4)}\\
\requirement{Wenn a = 0.5 ist und b = 0.4, dann ist das Schwellwertpaar
(0,4, 0.5)}\\
\requirement{Wenn a = b = 0.5 ist, dann ist das resultierende Schwellwertpaar
(0.5, 0.5)}\\
\requirement{Das Protokoll, indem beide Anwender ihre eigenen Schwellwerte
an den jeweils anderen Anwender senden und dann die bereits spezifizierte
Merge-Operation ausf\"uhren ist implementiert.}
\requirement{Wenn der E-Mail-Inhalt "A A A A B B C D" ist und die Attribute
sind (A, 0.2, 0.3), (B, 0.1, 0.9), (C, 0.5, 0.8),
dann wird diese E-Mail diskretisiert in den Vektor oft, mittel, selten}
\requirement{Dieser Schaltkeris f\"ur die dominierende Ausgabe 
kann generiert werden}
\requirement{Dieser Schaltkreis zur Bestimmung der dominierenden Ausgabe kann
generiert werden.}
\requirement{Es ist m\"oglich, einen Schaltkreis so zu erweitern (separate Ausgaben)}

\requirement{Es ist m\"oglich, einen Schaltkreis so zu erweitern (Shares)}
\requirement{Dieser Schaltkreis zur Berechnung der Shares der
ersten Approximation kann generiert werden}
\requirement{Dieses Protokoll zur Polynomauswertung kann ausgef\"uhrt werden.}
\requirement{Dieses Protokoll zum 1-out-of-N-Oblivious Transfer kann
ausgef\"uhrt werden.}

\requirement{Es ist so eine einweg Funktionsfamilie implementiert.}
\requirement{Dieses Protokoll zur Multiplikation kann ausgef\"uhrt werden}

\requirement{Die Entropie-Shares k\"onnen anhand diesse Protokolles berechnet
werden}
\requirement{Dieser Vergleichsschaltkreis kann generiert werden.}
\requirement{Dieses Protokoll kann implementiert werden}
\pagebreak
Nouns:
\begin{multicols}{2}
\begin{itemize}
\item Programm, Anwendung, Software
\item E-Mails
\item Verzeichnis
\item Benutzer, Anwender
\item Unterverzeichnis
\item Spam, Spam-Mails
\item Nicht Spam, Nicht-Spam-Mails
\item Ausgabeform
\item Klassifikator
\item Standardausgabe
\item Kommandozeilenparameter, Parameter
\item Dateiname
\item Ausgabedatei
\item Fehler, Problem
\item Server
\item Client
\item Berechtigungsproblem
\item Port
\item Fehlermeldung
\item IP Adresse, IP
\item Verbindung
\item Lernvorgang
\item Unerreichbarkeit
\item Phase
\item Tabelle
\item Spam-Anteil
\item Nicht-Spam-Anteil
\item Approximation
\item Wert
\item Fliesskommazahl
\item Wort
\item Vorkomniss
\item Wortliste
\item Protokoll
\item Syncronisierung
\item Mittelwert
\item Schwellwertpaar
\item Schwellwert
\item Merge-Operation
\item E-Mail Inhalt
\item Attribut
\item Vektor
\item Schaltkreis
\item Ausgabe
\item Share, Entropie-Share
\item Polynomauswertung
\item Polynom
\item 1-out-of-N-Oblivious-Transfer
\item FunktionsFamilie
\item Multiplikation
\item Vergleichsschaltkreis
\end{itemize}
\end{multicols}
\pagebreak
\section{Klassifikator}
\requirement{Der Klassifizierer weist die Eingabe Decide((Foo, 0.5,2),
Output(Spam), Output(Spam)) wegen eines zu grossen Schwellwertes zur\"uck}
\requirement{Der Klassifizierer weist die Eingabe Decide((Foo, 2, 3),
Output(Spam), Output(Spam)) wegen eines zu grossen Schwellwertes zur\"uck}
\requirement{Der Klassifizierer weist die Eingabe Decide((Foo, 1, 0),
Output(Spam), Output(Spam)) wegen falsch sortierter Schwellwerte zur\"uck}
\requirement{Der Klassifizierer weist die Eingabe Decide((Foo, 0.2, 0.3),
Output(Spam), Output(Spam)) wegen zuweniger Teilb\"aume zur\"uck}
\requirement{Der Klassifizierer weist die Eingabe Decide((Foo, 0.2, 0.3),
Output(Spam), Output(Spam), Output(Spam), Output(Spam)) wegen zuvieler
Teilb\"aume zur\"uck}
\requirement{Der Klassifizierer akzeptiert die Eingabe 
Decide((Foo, 0.2, 0.3), 
    Output(Spam),
    Decide((Bar, 0.3, 0.4)
        Output(Spam),
        Output(Not Spam), 
        Output(Spam)),
    Output(Spam))}
\requirement{Der Klassifizierer akzeptiert die Eingabe Decide((Foo, 0, 0.5),
Output(Spam), Output(Spam))}
\requirement{Der Klassifizierer akzeptiert die Eingabe Decide((Foo, 0.5, 1),
Output(Spam), Output(Spam))}
\requirement{Der Klassifizierer akzeptiert die Eingabe 
Decide((Foo, 0.5, 0.5)), Output(Spam), Output(Spam))}
\requirement{Der Klassifizierer akzeptiert die Eingabe Decide((Foo, 0, 0),
Output(Spam))}
\requirement{Der Klassifizierer akzeptiert die Eingabe Decide((Foo, 1, 1),
Output(Spam))}
\requirement{Gegeben der Klassifikator Decision((Bar, 0.3, 0.6), Output(Spam),
Output(Not Spam), Output(Spam)), dann wird die E-Mail "Bar Foo Foo Foo" als
Spam klassifiziert}
\requirement{Gegeben der Klassifikator Decision((Bar, 0.3, 0.6), Output(Spam),
Output(Not Spam), Output(Spam)), dann wird die E-Mail "Bar, Bar, Foo, Foo" als
Not Spam klassifiziert}
\requirement{Gegeben der Klassifikator Decision((Bar, 0.3, 0.6), Output(Spam),
Output(Not Spam), Output(Spam)), dann wird die E-Mail "Bar, Bar, Bar, Foo" als
Spam klassifiziert}
\requirement{Gegen der Klassifikator Decision((Bar, 0.3, 0.6), Output(Spam),
Output(Not Spam), Output(Spam)) und eine Verzeichnisstruktur wie oben
skizziert, wobei hank/mail1 "Bar Foo Foo Foo", hank/mail2 "Bar Bar Foo Foo"
und bob/mail1 "Bar Bar Bar Foo" enth\"alt, dann wird die oben als Beispiel
genannte Ausgabe produziert (oder in einer anderen Reihenfolge}

\pagebreak
\begin{multicols}{2}
\begin{itemize}
\item Klassifizierer
\item Eingabe
\item Schwellwert
\item Teilbaum
\item E-Mail
\item Spam
\item Nicht Spam
\item Verzeichnisstruktur
\item Beispiel
\item Ausgabe
\end{itemize}
\end{multicols}
\end{document}
