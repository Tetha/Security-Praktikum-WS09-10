\documentclass{article}

\begin{document}

\section{Aufgabe des Programmes}
  Das im Praktikum zu erstellende Programm hat die Aufgabe, aus 
den Wortlisten zweier beteiligter Parteien einen Klassifikator 
f\"ur Spammails zu erzeugen, den jede Partei einzeln verwenden kann.\\
  Es gibt bei der Verwendung des Programmes genau zwei beteiligte Parteien.
Diese werden im folgenden als Anwender bezeichnet. Jeder dieser beiden
Anwender muss fuer das Programm als Eingabe eine Liste von Worten und eine
Menge von E-Mail Inhalten bereitstellen.
  Diese Wortlisten enthalten einzelne Worte, welche der entsprechende
Anwender in der Spamklassifikation verwendet sehen will.
  Die Menge von Email Inhalten enth\"alt Inhalte von E-Mails, die gesammelt
und vorklassifiziert wurden.
  Der berechnete Klassifikator ist in der Lage, Mails in Spam und kein Spam
einzuteilen und kann von jedem Anwender ohne die anderen Anwender verwendet
werden.\\
  Dadurch ist es beispielsweise m\"oglich, dass Betreiber von E-Mail-Servern
gemeinsam bessere E-Mail-Klassifikatoren berechnen, als jeder einzelne mit
seinen vorklassifizierten E-Mails es koennte. Danach kann von jedem
Anwender der berechnete Klassifikator verwendet werden, um den weiteren 
E-Mailverkehr zu klassifizieren.\\

\section{Garantierter Level der Geheimhaltung}
  Das Programm garantiert in keiner Weise f\"ur die Geheimhaltung der
Wortliste. Es ist jedoch dem Anwender bekannt, dass die Wortliste publiziert
werden wird, sodass es jedem Anwender m\"oglich ist, keine geheimzuhaltenden
Informationen in die Wortliste zu \"ubernehmen.\\
  Das Programm garantiert nur daf\"ur, dass ueber die Menge der
vorklassifizierten E-Mail-Inhalte nur soviele Informationen preisgegeben
werden, wie aus dem resultierenden Klassifikator entnehmbar sind. Da
wir den ID3-Algorithmus verwenden, ist dies umformulierbar zu: Es wird 
potentiell f\"ur alle Worte der Wortliste eines einzelnen Anwenders 
preisgegeben, wie h\"aufig dieses Wort in der Vereinigung beider 
Mengen von E-Mail Inhalten vorkommt.
  Da jedoch jeder Benutzer selbst bestimmen kann, welche Worte in der 
Wortliste auftauchen, wurde dieser Informationsverlust als akzeptabel 
eingestuft, da er unvermeidbar ist.\\
  Die Publizierung der Wortliste ist unvermeidbar, da beide Parteien am
Ende einen Klassifikator erhalten, den sie alleine benutzen.
Das bedeutet, beide Parteien haben am Ende vollst\"andige Informationen
\"uber den Klassifikator.
  In unserem Fall wird ein Entscheidungsbaum berechnet. 
  Das bedeutet insbesondere, dass die Attribute des Baumes jedem Anwender
vollst\"andig bekannt sind.
  Da die Attribute des Baumes aus der Wortliste jedes Anwenders berechnet
werden, muss somit damit gerechnet werden, dass die vollst\"andige
Wortliste der Benutzer preisgegeben wird.\\
  Der Informationsverlust durch den Klassifikator kann ebenfalls nicht 
weiter reduziert werden.
  Es ist so, dass in der Lernphase eines Klassifikators Informationen aus
den Trainingsdaten des Klassifikators in den Klassifikator einkodiert
werden.
  Dies ist notwendig, damit der Klassifikator im folgenden ohne die
Trainingsmenge weiterhin klassifizieren kann.
  Damit und aus dem Fakt, dass jeder Anwender den Klassifikator alleine
vollst\"andig verwenden k\"onnen muss, folgt, dass jeder Informationen
\"uber die Gesamtmenge aller E-Mails enthalten wird, da dies durch die 
Natur eines Klassifikators und die vollst\"andigen Informationen \"uber
den Klassifikator impliziert wird.

\section{Sicherheit im \(ID3_\delta\)}
Die Sicherheit aller Bestandteile des Algorithmus' in dem Sinne, dass man \"uber
die Datenbank des anderen Anwenders nicht mehr erf\"ahrt als der Baum preisgibt,
ist im Paper bewiesen. Damit bleibt nur noch zu kl\"aren wie die Attribute
gefunden werden.

\section{Finden der Attribute des Baums}
Es gab 3 Ans\"atze zum Finden der Attribute:
\begin{itemize}
\item alle Worte aus den E-mail-Inhalten werden genutzt
\item das gesamte W\"orterbuch wird genutzt
\item beide Partner geben eine Wortliste vor
\end{itemize}

Bei Option 1 ist das Problem, dass der andere Anwender eine Vorstellung davon
gewinnt, welche Worte in den Nutzmails oft vorkommen, sodass die Geheimhaltung
der Inhalte der E-Mails st\"arker gef\"ahrdet ist.\\
Option 2 hat ein \"ahnliches Problem: Alle Worte im Lexikon sind mit grosser
Wahrscheinlichkeit eine Obermenge aller Worte, die in den E-Mails vorkommen. 
Damit ist analog zu Option 3 die Geheimhaltung der Inhalte der E-Mails st\"arker
gef\"ahrdet.\\
Bei Option 3 wird ebenfalls die Geheimhaltung der Inhalte der E-Mails 
gef\"ahrdet, jedoch hat jeder Anwender hier die direkte und uneingeschr\"ankte
Kontrolle \"uber diesen Verlust der Geheimhaltung.\\
Einen Informationsgewinn des anderen Anwenders kann man nicht verhindern,
da dieser alle Worte kennen muss um das n\"achste Attribut f\"ur den Baum zu 
bestimmen und durch das Lernen des Klassifikators auf jeden Fall genaues
Wissen dar\"uber erh\"alt, welche Attribute die Mails gut klassifizieren.\\
Generell gilt, dass die Wahl eines Attributes Informationen an den anderen
Anwender \"ubertr\"gt: mit hoher Wahrscheinlichkeit ist dieses Attribut,
welches gleichbedeutend mit einem Wort ist, f\"ur den anderen (auch) wichtig. 
Also k\"onnen wir nur vermeiden, dass ungewollt Informationen weitergegeben 
werden.\\
Wir definieren privat in dem Sinne dass jeder Anwender selbst 
entscheiden kann welche Informationen weitergegeben werden. 
Das fordert --- verglichen mit dem Paper --- weniger, sodass die Sicherheit des 
\(Id3_\delta\) nicht neu bewiesen werden muss.\\
Option 3 erf\"ullt diese Bedingung; wir gehen im Folgenden davon aus dass
beide eine Wortliste aufstellen.

\section{Finden der gemeinsamen Wortliste}
  Es bleibt zu kl\"aren, wie die Wortlisten zusammengef\"uhrt werden. 
  Dann sollte die preisgegebene Information \"uber die Wortliste minimal
sein, auch wenn im Folgenden demonstriert werden wird, dass keine
Geheimhaltung der Wortliste garantiert werden kann.\\
  Es seien oBdA. zwei Anwender beteiligt und \(W_1\) die Wortliste des
ersten Anwenders und \(W_2\) die Wortliste des zweiten Anwenders und 
\(join\) sei eine Operation die die beiden Wortlisten zusammenf\"uhrt.
  Dann sollten sowohl \(join(W_1, W_2) \subseteq W_1\) als auch
\(join(W_1, W_2) \subseteq W_2\) gelten.
  Damit ergibt sich der Schnitt der Wortlisten als einfache Implementierung
von \(join\).

\end{document}
