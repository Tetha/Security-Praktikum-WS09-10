\documentclass{article}

\usepackage{amsthm}

\newcommand{\defined}[1]{{\bf #1}}
\newcommand{\var}[1]{{\em #1}}

\theoremstyle{definition}
\newtheorem{definition}{Definition}

\theoremstyle{remark}
\newtheorem{remark}{Bemerkung}
\begin{document}

\begin{definition}
Anwender, aktiver Anwender\\
Die Benutzer des Programmes werden im folgenden als \defined{Anwender} bezeichnet.
  Im Folgendenden bezeichnen \var{A}, \var{B} zwei Anwender, die das Programm 
verwenden.\\
  Aus technischen Gr\"unden wird einer der beiden Anwender als 
\defined{aktiver Anwender} bezeichnet und der andere der beiden Anwender als
\defined{passiver Anwender}. Der aktive Anwender ist in der Lage, den Ablauf
eines Protokolles zu initiieren, w\"ahrend der passive Anwender auf die 
Initiierung eines Protokolles durch den aktiven Anwender wartet. Es sei im
folgenden \var{A} der aktive Anwender und \var{B} der passive Anwender.
\end{definition}

\begin{definition}
Wissen, Wissen eines Anwenders, gemeinsames Wissen.\\
Wenn \var{A} ein Anwender ist, so bezeichnen Ausdr\"ucke der Form 
\defined{\var{A.x}} f\"ur Variablen \var{x} \defined{Wissen des Anwenders}. 
Auf das Wissen des Anwenders kann nur der Anwender selbst zugreifen.
Das Wissen eines Anwenders kann \defined{\var{A.x} = \var{e}} erweitert werden.\\
Wenn ein Ausdruck \var{e} als \defined{gemeinsames Wissen c} bezeichnet wird, 
so ist dies \"aquivalent dazu, dass f\"ur alle beteiligten Anwender 
A \var{A.c} = \var{e} ist.
\end{definition}

\begin{definition}
Protokoll.\\
Wir bezeichnen im Folgenden eine Sequenz von NAchrichtenaustauschen zwischen
zwei Anwendern als \defined{Protokoll}
\end{definition}

\begin{definition}
Zwei-Parteien-Berechnung, private Zwei-Parteien-Berechnung, Eingabe, Ausgabe\\
Gegeben zwei Anwender \var{A}, \var{B} mit Wissen \var{A.i} und \var{B.i}
und eine Funktion \var{f}, dann wird ein Protokoll als
\defined{Zwei-Parteien-Berechnung} von Wissen j bezeichnet, wenn nach 
Ausf\"uhrung des Protokolls  \(A.j = f(A.i, B.i) \wedge B.j = f(A.i, B.i) \)
gilt. Wir bezeichnen i als \defined{Eingabe} des Protokolles der Anwender
und j als \defined{Ausgabe} des Protokolles.\\
Wenn ausserdem gilt, dass \var{A} nur \var{A.j} als Wissen ueber \var{B.i}
erh\"alt und dass \var{B} nur \var{B.j} als Wissen ueber \var{A.i} erh\"alt,
dann bezeichnen wir das Zwei-Parteien-Protokoll als \defined{privates
Zwei-Parteien-Protokoll}.
\end{definition}

\begin{remark}
Wenn ein Zwei-Parteien-Protokoll eine Funktion \var{f} berechnen soll,
die Eingaben \var{X} zus\"atzlich zum Wissen der Benutzer erfordert, 
kann dies durch Definition einer Funktion \var{f'(i, j) = f(x, i, j)}
erreicht werden. Dadurch k\"onnen beispielsweise gemeinsames Wissen oder
zuvor festzulegende Parameter genutzt werden. Damit ist die Definition 4
ausreichend, um beliebige Funktionen zu beschreiben.
\end{remark}

\end{document}
